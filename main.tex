\documentclass{article}

\usepackage[utf8x]{inputenc}
\usepackage[english,russian]{babel}

\usepackage{graphicx} % Required for inserting images
\usepackage{amsmath, amssymb, amsthm} 
\usepackage[letterpaper, top=1in, bottom=1.0in, left=1.20in, right=1.20in,heightrounded]{geometry}

\title{Mathematical Analysis Vol.1}
\author{@Souez3}
\date{22.11.2024}

\begin{document}

\maketitle

\section{Билет 1}
\subsection{Последовательность}

${f(n)}$ - последовательность задана на множестве N
Когда каждому $n\in \mathds{N}$ поставлено в соответствие некоторого закона
$a(n)\in \mathds{R}$, тогда говорят, что задана числовая последовательность ${a_n}^\inf$

Примеры: 
n-ный член арифметической прогрессии: $a_n = a_1 + \alpha(n-1)$
геометрическая прогрессия: $b_n = b_1 * q^(n-1)$

\subsection{Предел числовой последовательности}
\textbf{Определение:} Число А называют пределом числовой
последовательности ${X_n}$, если $\forall \epsilon > 0 \exists N(\epsilon) : \forall n > N(\epsilon)$ выполняется $|X_n-A| < \epsilon$

\textbf{Определение:} Сходящаяся последовательность - последовательность, которая имеет конечный предел

\textbf{Определение:} Расходящаяся последовательность - последовательность, которая имеет бесконечный предел либо предела не существует.

Последовательноть ограничена, если $\exists M > 0 : \forall n \in \mathds{N}$ выполняется ${a_n} <= M$
(существует такое число М, что для любого номера последовательности все члены последовательности не превосходят это число по модулю.

\section{Билет 2}
\subsection{Теорема о единственности предела последовательности}

\textbf{Теорема:} Если у последовательности есть предел, то он единственный

\textbf{Доказательство:} 
Докажем от противного. Допустим существует 2 предела.
\begin{equation}
    $\sqsupset \lim_{x \to \infty}{X_n} = A$
    $\sqsupset \lim_{x \to \infty}{X_n} = B$, при этом $B!=A$
\end{equation}

Тогда возьмем $\epsilon = (B-A)/3 > 0$, $(\epsilon_A \bigcap \epsilon_B != 0)$

Следовательно
\begin{equation}
    Для $n >= N \exists номер N_1 : \forall n > N$ выполняется $|{X_n} - A| < \epsilon$
\end{equation}
 
\begin{equation}
    Также $\exists N_2 для \forall_n >= N_2$ и тоже выполняется, что $|{X_n} - B| < \epsilon$
\end{equation}
Тогда $|a - b| = |a - {X_n} + {X_n} - b| <= |{X_n} - A| + {X_n} - B| < \epsilon + \epsilon = 2\epsilon = \frac{2*|A-B|}{3}$, тогда получим $|A-B| <= \frac{2}{3}*|B-A|$ 
Получим противорчие

\section{Билет 3}

\textbf{Определение:} Последовательность ограничена, если $\exists M > 0 : \forall b \in \mathds{N}$ выполняется $|a_n| <= M$

\textbf{Теорема об ограниченности сходящейся последовательности:} Всякая сходящаяся последовательность ограничена!

\textbf{Доказательство:} $\sqsupset A = \lim_{n \to \infty}{X_n} \in \mathds{R}$, тогда и только тогда, когда $\forall \epsilon > 0 \exists N(\epsilon) \in mathds{N}$ такое что $\forall n \in mathds{N} : n > N(\epsilon)$ выполняется $|{X_n} - A| < \epsilon \forall n > N(\epsilon) {X_n} \in (A-\epsilon; A+\epsilon)$ содержит конечное число x_1,x_2,...x_k
$\sqsupset m = min{X^-; A-\epsilon} и M = max{A-\epsilon; x^+}$ 
Тогда на отрезке [m;M] находятся x_1,x_2,...x_k и (A-\epsilon;A+\epsilon) следовательно [m;M] содержат все точки {x_n}, то $\forall n \in mathds{N} x_n<=m x_n>=M$

Примеры:
\begin{equation}
    1) ${\frac{1}{n^2}} = {1; \frac{1}{4}; \frac{1}{4}; \frac{1}{9}; \frac{1}{16}...}$ $\lim{\frac{1}{n^2}} = 0$ - ограничена сверху
    2) ${\frac{n^2}{n+1}} = {\frac{1}{2}; \frac{4}{3}; \frac{9}{4}; \frac{16}{5};...}$ $\lim{\frac{n^2}{n+1}} >= \frac{1}{2}$ - ограничена снизу
\end{equation}

\section{Билет 4}
Арифметические операции над сходящимися последовательностями

$\sqsupset{X_n}; {Y_n}$ - две сходящиеся последовательности. Тогда $\exists \lim_{n \to \infty}{X_n} = A; \lim_{n \to \infty}{Y_n} = B$

Cвойства
1) ${X_n += Y_n}; {{X_n} * {Y_n}}; {\frac{X_n}{Y_n}}$ - тоже сходящиеся последовательности.
2) $\lim_{n \to \infty}({X_n} + {Y_n}) = A+B$
3) $\lim_{n \to \infty}({X_n} - {Y_n}) = A-B$
4) $\lim_{n \to \infty}({X_n} * {Y_n}) = A*B$
5) $\lim_{n \to \infty}\frac{X_n}{Y_n} = \frac{A}{B}$

\textbf{Доказательство:} 
1) $\forall N > 0 \existsN_0 : \forall n > N_0$ выполняется $|X_n - A| < \frac{\epsilon}{2}(любая окрестность)$
$\exists N_1 : \forall n > N_1$ выполняется $|{Y_n}-B| < \frac{\epsilon}{2}$
Пусть N = max(N_2; N_1), n > N
Тогда $\forall n > N |({X_n}+{Y_n}) - (A+B)| = |{X_n}- A + {Y_n} - B| <= |{X_n}- A| + |{Y_n} - B| < \frac{\epsilon}{2} + \frac{\epsilon}{2} = \epsilon$

\section{Билет 5}
\subsection{Понятие функции через последовательность}

Если каждому $x \in X$ по некоторому закону поставлен в соответствии единственный y, то говорят что на множестве X задана функция f
\begin{equation}
    $\forall x \in X \exists! y \in \mathds{R} : f(x) = y$
\end{equation}

\subsection{Предел функции в точке}
\textbf{Определение по Гейне:} $\sqsupset f(x)$ - определена в некоторой проколотой окрестности точки x
\begin{equation}
    $\lim _{x \to x_0}{f(x)} = A$ если $\forall {x_n} \exists \mathring{U}_x_0 > 0$

    $\lim _{x \to x_0}{f(x) - g(x)} > 0$ => $f(x) - g(x) > 0$ по теореме если f(x) имеет предел A и в окрестности (а) принимает значения больше нуля, то A >= 0
\end{equation}

\subsection{Теорема о единственности предела} 
Если функция имеет предел в точке, то он единственнй.

Доказательство от противного:
$\sqsupset \exists {X_n} = \lim _{n \to \infty}{X_n} = A$ и $\lim _{n \to \infty}{X_n} = B$, $A != B; A,B \in \mathds{R}$
Возьмем $\epsilon_n \bigcap \epsilon_b != \empty$, тогда $|f(x) - A| < \frac{\epsilon}{2}; |f(x) - B| < \frac{\epsilon}{2}$
$|A-B| = |A-B+f(x)-f(x)| = |A-f(x)+f(x)-B| <= |A-f(x)| + |B-f(x)| < \frac{\epsilon}{2} + \frac{\epsilon}{2} = \epsilon$
То есть получили $\forall \epsilon > 0 -> |A-B| < \epsilon$

\section{Билет 6}

\subsection{Ограниченная функция}

\textbf{Определение:} Последовательность ограничена, если $\exists M > 0 : \forall b \in \mathds{N}$ выполняется $|a_n| <= M$

\textbf{Определение:} Функция называется ограниченной сверху на x если $\exists M : \forall x \in X$ выполняется $F(x)<M$, снизу соответсвенно $F(x)>M$

\textbf{Определение:} Теорема об ограниченности функции, имеющей предел (конечный)
Если функция $f(x)$ определена в точке $x_0$ и имеет в точке конечный предел, то она ограничена в некоторой окрестности этой точки.

\[
\exists \lim_{x \to x_0} f(x) = A \iff \forall \varepsilon > 0 \, \exists \delta > 0 : \forall x \in \dot{U}(\delta),
\]
\[
|x - x_0| < \delta \implies |f(x) - A| < \varepsilon.
\]

Пусть $\varepsilon = 1$, тогда $\forall x \in \dot{U}(\delta)$:

\[
|f(x) - A| < 1,
\]

раскрыв модуль:

\[
-1 < f(x) - A < 1.
\]

Отсюда:

\[
A - 1 < f(x) < A + 1 \implies f(x) \text{ ограничена}.
\]

\section{Билет 7}

\subsection{Арифметические действия с пределами функции} 

\[
\lim_{x \to x_0} f(x) = A \quad \text{и} \quad \lim_{x \to x_0} \varphi(x) = B.
\]

Тогда:

1. \[
\lim_{x \to x_0} (f(x) + \varphi(x)) = A + B.
\]

2. \[
\lim_{x \to x_0} C \cdot f(x) = C \cdot A, \quad \text{где } C = \text{const}.
\]

3. \[
\lim_{x \to x_0} (f(x) \cdot \varphi(x)) = A \cdot B.
\]

4. Если $B \neq 0$, то:

\[
\lim_{x \to x_0} \frac{f(x)}{\varphi(x)} = \frac{A}{B}.
\]

Условие: $\forall x \in \operatorname{Dom}(\varphi) \quad \varphi(x) \neq 0$.

\subsection*{Доказательство: Арифметическое свойство предела (Сумма)}

\subsection*{Условие}
\[
\lim_{x \to x_0} f(x) = A \quad \text{и} \quad \lim_{x \to x_0} \varphi(x) = B.
\]

\subsection*{Доказательство}
По определению предела:

\[
\lim_{x \to x_0} f(x) = A \iff \forall \varepsilon_1 > 0 \, \exists \delta_1 > 0 : \forall x \in \dot{U}(\delta_1),
\]
\[
|x - x_0| < \delta_1 \implies |f(x) - A| < \varepsilon_1.
\]

\[
\lim_{x \to x_0} \varphi(x) = B \iff \forall \varepsilon_2 > 0 \, \exists \delta_2 > 0 : \forall x \in \dot{U}(\delta_2),
\]
\[
|x - x_0| < \delta_2 \implies |\varphi(x) - B| < \varepsilon_2.
\]

Пусть $\varepsilon = \varepsilon_1 + \varepsilon_2$, и $\delta = \min(\delta_1, \delta_2)$. Тогда:

\[
|f(x) + \varphi(x) - (A + B)| = |f(x) - A + \varphi(x) - B| \leq |f(x) - A| + |\varphi(x) - B|.
\]

Из условий следует:

\[
|f(x) - A| < \varepsilon_1 \quad \text{и} \quad |\varphi(x) - B| < \varepsilon_2.
\]

Таким образом:

\[
|f(x) + \varphi(x) - (A + B)| < \varepsilon_1 + \varepsilon_2 = \varepsilon.
\]

\subsection*{Вывод}
\[
\lim_{x \to x_0} (f(x) + \varphi(x)) = A + B.
\]

\subsection*{Теорема о суперпозиции}

1) \( f(x) и g(x) \): \( F(x) = F(f(g(x))) \)

2) \( \lim_{x \to x_0} g(x) = A \)

3) \( \lim_{x \to x_0} f(x) = B \)

Следовательно:
\[
\lim_{x \to x_0} F(f(g(x))) = B
\]

\section{Билет 8}

\subsection*{Теоремы о пределах функции: о предельном переходе в неравенство}

Рассмотрим неравенство:
\[
a_n \leq b_n
\]

Пусть \( \lim_{n \to \infty} a_n = A \) и \( \lim_{n \to \infty} b_n = B \). Тогда, если \( a_n \leq b_n \) для всех \( n \), то по свойству пределов:
\[
\lim_{n \to \infty} a_n \leq \lim_{n \to \infty} b_n
\]

Следовательно:
\[
A \leq B
\]

\subsection*{Теорема о сжатой функции}
Пусть \( f(x) \), \( g(x) \) и \( h(x) \) — функции, определенные на множестве \( E \subset \mathbb{R} \) и выполняется неравенство
\[
f(x) \leq h(x) \leq g(x),
\]
и при этом
\[
\lim_{x \to a} f(x) = \lim_{x \to a} g(x) = b,
\]
то
\[
\lim_{x \to a} h(x) = b.
\]

\subsection*{1 замечательный предел}
Рассмотрим предел:
\[
\lim_{x \to 0} \frac{\sin x}{x} = 1
\]

\textbf{Доказательство:}

Рассмотрим односторонние пределы и докажем, что они равны 1. Рассмотрим случай \( x \to +0 \). Отложим этот угол на единичной окружности так, чтобы его вершина совпадала с началом координат, а одна сторона совпадала с осью \( OX \). Пусть \( A \) — точка пересечения второй стороны угла с единичной окружностью, а точка \( B \) — с касательной к этой окружности в точке \( A \). Точка \( C \) — проекция точки \( A \) на ось \( OX \). Очевидно, что:
\[
S_{\triangle OAC} < S_{\text{сектора } OAC} < S_{\triangle OAB}
\]
где \( S \) — площадь. Поскольку \( |OC| = \cos x \), \( |AC| = \sin x \), \( |AB| = \tan x \), то:
\[
\frac{\sin x}{2} < \frac{x}{2} < \frac{\tan x}{2}
\]
Так как при \( x \to +0 \): \( \sin x > 0 \), \( x > 0 \), \( \tan x > 0 \):
\[
\frac{1}{\tan x} < \frac{1}{x} < \frac{1}{\sin x}
\]
Умножаем на \( \sin x \):
\[
\cos x \leq \frac{\sin x}{x} \leq 
\]
Переходя к пределу:
\[
\lim_{x \to +0} \cos x \leq \lim_{x \to +0} \frac{\sin x}{x} \leq 1
\]
Так как \( \lim_{x \to +0} \cos x = 1 \), то:
\[
\lim_{x \to +0} \frac{\sin x}{x} = 1
\]

Аналогично доказывается для \( x \to -0 \). Следовательно:
\[
\lim_{x \to 0} \frac{\sin x}{x} = 1
\]

\section{Билет 9}

\subsection*{Предел функции на бесконечности}
Будем говорить, что на бесконечности функция \( f(x) \) стремится к пределу \( A \), если \( \forall \varepsilon > 0 \exists X > M \) такое, что \( |x| > X \Rightarrow |f(x) - A| < \varepsilon \).

\section{Билет 10}

\subsection*{Бесконечно большие функции.}

Функция \( f(x) \) называется бесконечно большой при \( x \to x_0 \), если
\[
\forall M > 0 \, \exists \delta > 0 \, \text{такое, что} \, 0 < |x - x_0| < \delta \Rightarrow |f(x)| > M
\]

Пример:
Функция \( f(x) = \frac{1}{x} \) является бесконечно большой при \( x \to 0 \).


\section{Билет 20}

\subsection*{Устойчивость знака непрерывной функции}

\textbf{Теорема:} Пусть $f$ — непрерывная функция на множестве $D \subset \mathbb{R}$, и пусть $c \in D$ такая точка, что $f(c) \neq 0$. Тогда существует окрестность $U(c)$ точки $c$, такая что для всех $x \in U(c) \cap D$ выполняется $f(x) \neq 0$ и знак функции $f$ на $U(c) \cap D$ совпадает со знаком $f(c)$.


\section{Билет 21}

\subsection*{1. Алгебраические функции}
Алгебраические функции — это функции, которые могут быть выражены с использованием конечного числа операций сложения, вычитания, умножения, деления и извлечения корней. Примеры:
\begin{itemize}
    \item линейная функция: $f(x) = ax + b$, где $a, b \in \mathbb{R}$;
    \item квадратичная функция: $f(x) = ax^2 + bx + c$, где $a, b, c \in \mathbb{R}$;
    \item корневая функция: $f(x) = \sqrt[n]{x}$, где $n \in \mathbb{N}, n \geq 2$.
\end{itemize}

\subsection*{2. Трансцендентные функции}
Трансцендентные функции не могут быть выражены в виде конечных комбинаций алгебраических операций. Они включают:
\begin{itemize}
    \item экспоненциальные функции, например, $f(x) = a^x$, где $a > 0, a \neq 1$;
    \item логарифмические функции, например, $f(x) = \ln(x)$ или $f(x) = \log_a(x)$;
    \item тригонометрические функции: $\sin(x)$, $\cos(x)$, $\tan(x)$ и т.д.;
    \item обратные тригонометрические функции: $\arcsin(x)$, $\arccos(x)$ и т.д.;
    \item гиперболические функции: $\sinh(x)$, $\cosh(x)$ и т.д.
\end{itemize}

\section{Билет 21}

\subsection*{3. Непрерывность элементарных функций}
Элементарные функции являются непрерывными на своих областях определения. Это означает, что если функция определена в некоторой точке $x_0$ и в её окрестности, то:
\[
\lim_{x \to x_0} f(x) = f(x_0).
\]
Примеры:
\begin{itemize}
    \item Линейные и квадратичные функции непрерывны на всей числовой прямой $\mathbb{R}$.
    \item Тригонометрические функции $\sin(x)$ и $\cos(x)$ непрерывны на $\mathbb{R}$, а $\tan(x)$ — на множестве $\mathbb{R} \setminus \left\{x = \frac{\pi}{2} + k\pi, \, k \in \mathbb{Z}\right\}$.
\end{itemize}

\section*{Операции над непрерывными функциями и переход к пределу под знаком непрерывной функции}

\subsection*{Операции над непрерывными функциями}
Пусть $f$ и $g$ — функции, непрерывные в точке $x = a$. Тогда следующие функции также непрерывны в точке $a$:
\begin{itemize}
    \item Сумма: $f(x) + g(x)$
    \item Разность: $f(x) - g(x)$
    \item Произведение: $f(x) \cdot g(x)$
    \item Частное: $\frac{f(x)}{g(x)}$, если $g(a) \neq 0$
\end{itemize}

\subsection*{Переход к пределу под знаком непрерывной функции}
Пусть $f$ — непрерывная функция в точке $a$, и пусть $\lim_{x \to a} g(x) = L$. Тогда:
\[
\lim_{x \to a} f(g(x)) = f\left(\lim_{x \to a} g(x)\right) = f(L)
\]

\textbf{Доказательство:} Так как $f$ непрерывна в точке $L$, то по определению непрерывности для любого $\epsilon > 0$ существует $\delta > 0$ такое, что для всех $y$, удовлетворяющих условию $|y - L| < \delta$, выполняется $|f(y) - f(L)| < \epsilon$. Поскольку $\lim_{x \to a} g(x) = L$, существует такое $\delta' > 0$, что для всех $x$, удовлетворяющих условию $|x - a| < \delta'$, выполняется $|g(x) - L| < \delta$. Следовательно, для таких $x$ имеем:
\[
|f(g(x)) - f(L)| < \epsilon
\]
Таким образом, $\lim_{x \to a} f(g(x)) = f(L)$.

\section{Билет 22}

\section*{Теорема о непрерывности сложной функции}

\textbf{Теорема:} Пусть $f$ непрерывна в точке $a$, и $g$ непрерывна в точке $b = f(a)$. Тогда сложная функция $h(x) = g(f(x))$ непрерывна в точке $a$.

\textbf{Доказательство:} Так как $f$ непрерывна в точке $a$, то для любого $\epsilon > 0$ существует $\delta_1 > 0$ такое, что если $|x - a| < \delta_1$, то $|f(x) - f(a)| < \delta_2$, где $\delta_2$ будет определено далее.

Поскольку $g$ непрерывна в точке $b = f(a)$, то для любого $\epsilon > 0$ существует $\delta_2 > 0$ такое, что если $|y - f(a)| < \delta_2$, то $|g(y) - g(f(a))| < \epsilon$.

Теперь, выберем $\delta = \delta_1$. Тогда, если $|x - a| < \delta$, то $|f(x) - f(a)| < \delta_2$, и, следовательно, $|g(f(x)) - g(f(a))| < \epsilon$.

Таким образом, $|h(x) - h(a)| = |g(f(x)) - g(f(a))| < \epsilon$, что доказывает непрерывность $h(x)$ в точке $a$.


\section{Билет 23}
\subsection*{Первый замечательный предел}
\[
\lim_{x \to 0} \frac{\sin x}{x} = 1
\]

\textbf{Следствия:}
\[
\lim_{x \to 0} \frac{\tan x}{x} = 1, \quad \lim_{x \to 0} \frac{\arcsin x}{x} = 1, \quad \lim_{x \to 0} \frac{\arctan x}{x} = 1
\]
\[
\lim_{x \to 0} \frac{1 - \cos x}{x^2/2} = 1
\]

\subsection*{Второй замечательный предел}
\[
\lim_{x \to \infty} \left(1 + \frac{1}{x}\right)^x = e
\]

\textbf{Следствия:}
\documentclass{article}

\usepackage[utf8x]{inputenc}
\usepackage[english,russian]{babel}

\usepackage{graphicx} % Required for inserting images
\usepackage{amsmath, amssymb, amsthm} 
\usepackage[letterpaper, top=1in, bottom=1.0in, left=1.20in, right=1.20in,heightrounded]{geometry}

\title{Mathematical Analysis Vol.1}
\author{@Souez3}
\date{22.11.2024}

\begin{document}

\maketitle

\section{Билет 1}
\subsection{Последовательность}

${f(n)}$ - последовательность задана на множестве N
Когда каждому $n\in \mathds{N}$ поставлено в соответствие некоторого закона
$a(n)\in \mathds{R}$, тогда говорят, что задана числовая последовательность ${a_n}^\inf$

Примеры: 
n-ный член арифметической прогрессии: $a_n = a_1 + \alpha(n-1)$
геометрическая прогрессия: $b_n = b_1 * q^(n-1)$

\subsection{Предел числовой последовательности}
\textbf{Определение:} Число А называют пределом числовой
последовательности ${X_n}$, если $\forall \epsilon > 0 \exists N(\epsilon) : \forall n > N(\epsilon)$ выполняется $|X_n-A| < \epsilon$

\textbf{Определение:} Сходящаяся последовательность - последовательность, которая имеет конечный предел

\textbf{Определение:} Расходящаяся последовательность - последовательность, которая имеет бесконечный предел либо предела не существует.

Последовательноть ограничена, если $\exists M > 0 : \forall n \in \mathds{N}$ выполняется ${a_n} <= M$
(существует такое число М, что для любого номера последовательности все члены последовательности не превосходят это число по модулю.

\section{Билет 2}
\subsection{Теорема о единственности предела последовательности}

\textbf{Теорема:} Если у последовательности есть предел, то он единственный

\textbf{Доказательство:} 
Докажем от противного. Допустим существует 2 предела.
\begin{equation}
    $\sqsupset \lim_{x \to \infty}{X_n} = A$
    $\sqsupset \lim_{x \to \infty}{X_n} = B$, при этом $B!=A$
\end{equation}

Тогда возьмем $\epsilon = (B-A)/3 > 0$, $(\epsilon_A \bigcap \epsilon_B != 0)$

Следовательно
\begin{equation}
    Для $n >= N \exists номер N_1 : \forall n > N$ выполняется $|{X_n} - A| < \epsilon$
\end{equation}
 
\begin{equation}
    Также $\exists N_2 для \forall_n >= N_2$ и тоже выполняется, что $|{X_n} - B| < \epsilon$
\end{equation}
Тогда $|a - b| = |a - {X_n} + {X_n} - b| <= |{X_n} - A| + {X_n} - B| < \epsilon + \epsilon = 2\epsilon = \frac{2*|A-B|}{3}$, тогда получим $|A-B| <= \frac{2}{3}*|B-A|$ 
Получим противорчие

\section{Билет 3}

\textbf{Определение:} Последовательность ограничена, если $\exists M > 0 : \forall b \in \mathds{N}$ выполняется $|a_n| <= M$

\textbf{Теорема об ограниченности сходящейся последовательности:} Всякая сходящаяся последовательность ограничена!

\textbf{Доказательство:} $\sqsupset A = \lim_{n \to \infty}{X_n} \in \mathds{R}$, тогда и только тогда, когда $\forall \epsilon > 0 \exists N(\epsilon) \in mathds{N}$ такое что $\forall n \in mathds{N} : n > N(\epsilon)$ выполняется $|{X_n} - A| < \epsilon \forall n > N(\epsilon) {X_n} \in (A-\epsilon; A+\epsilon)$ содержит конечное число x_1,x_2,...x_k
$\sqsupset m = min{X^-; A-\epsilon} и M = max{A-\epsilon; x^+}$ 
Тогда на отрезке [m;M] находятся x_1,x_2,...x_k и (A-\epsilon;A+\epsilon) следовательно [m;M] содержат все точки {x_n}, то $\forall n \in mathds{N} x_n<=m x_n>=M$

Примеры:
\begin{equation}
    1) ${\frac{1}{n^2}} = {1; \frac{1}{4}; \frac{1}{4}; \frac{1}{9}; \frac{1}{16}...}$ $\lim{\frac{1}{n^2}} = 0$ - ограничена сверху
    2) ${\frac{n^2}{n+1}} = {\frac{1}{2}; \frac{4}{3}; \frac{9}{4}; \frac{16}{5};...}$ $\lim{\frac{n^2}{n+1}} >= \frac{1}{2}$ - ограничена снизу
\end{equation}

\section{Билет 4}
Арифметические операции над сходящимися последовательностями

$\sqsupset{X_n}; {Y_n}$ - две сходящиеся последовательности. Тогда $\exists \lim_{n \to \infty}{X_n} = A; \lim_{n \to \infty}{Y_n} = B$

Cвойства
1) ${X_n += Y_n}; {{X_n} * {Y_n}}; {\frac{X_n}{Y_n}}$ - тоже сходящиеся последовательности.
2) $\lim_{n \to \infty}({X_n} + {Y_n}) = A+B$
3) $\lim_{n \to \infty}({X_n} - {Y_n}) = A-B$
4) $\lim_{n \to \infty}({X_n} * {Y_n}) = A*B$
5) $\lim_{n \to \infty}\frac{X_n}{Y_n} = \frac{A}{B}$

\textbf{Доказательство:} 
1) $\forall N > 0 \existsN_0 : \forall n > N_0$ выполняется $|X_n - A| < \frac{\epsilon}{2}(любая окрестность)$
$\exists N_1 : \forall n > N_1$ выполняется $|{Y_n}-B| < \frac{\epsilon}{2}$
Пусть N = max(N_2; N_1), n > N
Тогда $\forall n > N |({X_n}+{Y_n}) - (A+B)| = |{X_n}- A + {Y_n} - B| <= |{X_n}- A| + |{Y_n} - B| < \frac{\epsilon}{2} + \frac{\epsilon}{2} = \epsilon$

\section{Билет 5}
\subsection{Понятие функции через последовательность}

Если каждому $x \in X$ по некоторому закону поставлен в соответствии единственный y, то говорят что на множестве X задана функция f
\begin{equation}
    $\forall x \in X \exists! y \in \mathds{R} : f(x) = y$
\end{equation}

\subsection{Предел функции в точке}
\textbf{Определение по Гейне:} $\sqsupset f(x)$ - определена в некоторой проколотой окрестности точки x
\begin{equation}
    $\lim _{x \to x_0}{f(x)} = A$ если $\forall {x_n} \exists \mathring{U}_x_0 > 0$

    $\lim _{x \to x_0}{f(x) - g(x)} > 0$ => $f(x) - g(x) > 0$ по теореме если f(x) имеет предел A и в окрестности (а) принимает значения больше нуля, то A >= 0
\end{equation}

\subsection{Теорема о единственности предела} 
Если функция имеет предел в точке, то он единственнй.

Доказательство от противного:
$\sqsupset \exists {X_n} = \lim _{n \to \infty}{X_n} = A$ и $\lim _{n \to \infty}{X_n} = B$, $A != B; A,B \in \mathds{R}$
Возьмем $\epsilon_n \bigcap \epsilon_b != \empty$, тогда $|f(x) - A| < \frac{\epsilon}{2}; |f(x) - B| < \frac{\epsilon}{2}$
$|A-B| = |A-B+f(x)-f(x)| = |A-f(x)+f(x)-B| <= |A-f(x)| + |B-f(x)| < \frac{\epsilon}{2} + \frac{\epsilon}{2} = \epsilon$
То есть получили $\forall \epsilon > 0 -> |A-B| < \epsilon$

\section{Билет 6}

\subsection{Ограниченная функция}

\textbf{Определение:} Последовательность ограничена, если $\exists M > 0 : \forall b \in \mathds{N}$ выполняется $|a_n| <= M$

\textbf{Определение:} Функция называется ограниченной сверху на x если $\exists M : \forall x \in X$ выполняется $F(x)<M$, снизу соответсвенно $F(x)>M$

\textbf{Определение:} Теорема об ограниченности функции, имеющей предел (конечный)
Если функция $f(x)$ определена в точке $x_0$ и имеет в точке конечный предел, то она ограничена в некоторой окрестности этой точки.

\[
\exists \lim_{x \to x_0} f(x) = A \iff \forall \varepsilon > 0 \, \exists \delta > 0 : \forall x \in \dot{U}(\delta),
\]
\[
|x - x_0| < \delta \implies |f(x) - A| < \varepsilon.
\]

Пусть $\varepsilon = 1$, тогда $\forall x \in \dot{U}(\delta)$:

\[
|f(x) - A| < 1,
\]

раскрыв модуль:

\[
-1 < f(x) - A < 1.
\]

Отсюда:

\[
A - 1 < f(x) < A + 1 \implies f(x) \text{ ограничена}.
\]

\section{Билет 7}

\subsection{Арифметические действия с пределами функции} 

\[
\lim_{x \to x_0} f(x) = A \quad \text{и} \quad \lim_{x \to x_0} \varphi(x) = B.
\]

Тогда:

1. \[
\lim_{x \to x_0} (f(x) + \varphi(x)) = A + B.
\]

2. \[
\lim_{x \to x_0} C \cdot f(x) = C \cdot A, \quad \text{где } C = \text{const}.
\]

3. \[
\lim_{x \to x_0} (f(x) \cdot \varphi(x)) = A \cdot B.
\]

4. Если $B \neq 0$, то:

\[
\lim_{x \to x_0} \frac{f(x)}{\varphi(x)} = \frac{A}{B}.
\]

Условие: $\forall x \in \operatorname{Dom}(\varphi) \quad \varphi(x) \neq 0$.

\subsection*{Доказательство: Арифметическое свойство предела (Сумма)}

\subsection*{Условие}
\[
\lim_{x \to x_0} f(x) = A \quad \text{и} \quad \lim_{x \to x_0} \varphi(x) = B.
\]

\subsection*{Доказательство}
По определению предела:

\[
\lim_{x \to x_0} f(x) = A \iff \forall \varepsilon_1 > 0 \, \exists \delta_1 > 0 : \forall x \in \dot{U}(\delta_1),
\]
\[
|x - x_0| < \delta_1 \implies |f(x) - A| < \varepsilon_1.
\]

\[
\lim_{x \to x_0} \varphi(x) = B \iff \forall \varepsilon_2 > 0 \, \exists \delta_2 > 0 : \forall x \in \dot{U}(\delta_2),
\]
\[
|x - x_0| < \delta_2 \implies |\varphi(x) - B| < \varepsilon_2.
\]

Пусть $\varepsilon = \varepsilon_1 + \varepsilon_2$, и $\delta = \min(\delta_1, \delta_2)$. Тогда:

\[
|f(x) + \varphi(x) - (A + B)| = |f(x) - A + \varphi(x) - B| \leq |f(x) - A| + |\varphi(x) - B|.
\]

Из условий следует:

\[
|f(x) - A| < \varepsilon_1 \quad \text{и} \quad |\varphi(x) - B| < \varepsilon_2.
\]

Таким образом:

\[
|f(x) + \varphi(x) - (A + B)| < \varepsilon_1 + \varepsilon_2 = \varepsilon.
\]

\subsection*{Вывод}
\[
\lim_{x \to x_0} (f(x) + \varphi(x)) = A + B.
\]

\subsection*{Теорема о суперпозиции}

1) \( f(x) и g(x) \): \( F(x) = F(f(g(x))) \)

2) \( \lim_{x \to x_0} g(x) = A \)

3) \( \lim_{x \to x_0} f(x) = B \)

Следовательно:
\[
\lim_{x \to x_0} F(f(g(x))) = B
\]

\section{Билет 8}

\subsection*{Теоремы о пределах функции: о предельном переходе в неравенство}

Рассмотрим неравенство:
\[
a_n \leq b_n
\]

Пусть \( \lim_{n \to \infty} a_n = A \) и \( \lim_{n \to \infty} b_n = B \). Тогда, если \( a_n \leq b_n \) для всех \( n \), то по свойству пределов:
\[
\lim_{n \to \infty} a_n \leq \lim_{n \to \infty} b_n
\]

Следовательно:
\[
A \leq B
\]

\subsection*{Теорема о сжатой функции}
Пусть \( f(x) \), \( g(x) \) и \( h(x) \) — функции, определенные на множестве \( E \subset \mathbb{R} \) и выполняется неравенство
\[
f(x) \leq h(x) \leq g(x),
\]
и при этом
\[
\lim_{x \to a} f(x) = \lim_{x \to a} g(x) = b,
\]
то
\[
\lim_{x \to a} h(x) = b.
\]

\subsection*{1 замечательный предел}
Рассмотрим предел:
\[
\lim_{x \to 0} \frac{\sin x}{x} = 1
\]

\textbf{Доказательство:}

Рассмотрим односторонние пределы и докажем, что они равны 1. Рассмотрим случай \( x \to +0 \). Отложим этот угол на единичной окружности так, чтобы его вершина совпадала с началом координат, а одна сторона совпадала с осью \( OX \). Пусть \( A \) — точка пересечения второй стороны угла с единичной окружностью, а точка \( B \) — с касательной к этой окружности в точке \( A \). Точка \( C \) — проекция точки \( A \) на ось \( OX \). Очевидно, что:
\[
S_{\triangle OAC} < S_{\text{сектора } OAC} < S_{\triangle OAB}
\]
где \( S \) — площадь. Поскольку \( |OC| = \cos x \), \( |AC| = \sin x \), \( |AB| = \tan x \), то:
\[
\frac{\sin x}{2} < \frac{x}{2} < \frac{\tan x}{2}
\]
Так как при \( x \to +0 \): \( \sin x > 0 \), \( x > 0 \), \( \tan x > 0 \):
\[
\frac{1}{\tan x} < \frac{1}{x} < \frac{1}{\sin x}
\]
Умножаем на \( \sin x \):
\[
\cos x \leq \frac{\sin x}{x} \leq 
\]
Переходя к пределу:
\[
\lim_{x \to +0} \cos x \leq \lim_{x \to +0} \frac{\sin x}{x} \leq 1
\]
Так как \( \lim_{x \to +0} \cos x = 1 \), то:
\[
\lim_{x \to +0} \frac{\sin x}{x} = 1
\]

Аналогично доказывается для \( x \to -0 \). Следовательно:
\[
\lim_{x \to 0} \frac{\sin x}{x} = 1
\]

\section{Билет 9}

\subsection*{Предел функции на бесконечности}
Будем говорить, что на бесконечности функция \( f(x) \) стремится к пределу \( A \), если \( \forall \varepsilon > 0 \exists X > M \) такое, что \( |x| > X \Rightarrow |f(x) - A| < \varepsilon \).

\section{Билет 10}

\subsection*{Бесконечно большие функции.}

Функция \( f(x) \) называется бесконечно большой при \( x \to x_0 \), если
\[
\forall M > 0 \, \exists \delta > 0 \, \text{такое, что} \, 0 < |x - x_0| < \delta \Rightarrow |f(x)| > M
\]

Пример:
Функция \( f(x) = \frac{1}{x} \) является бесконечно большой при \( x \to 0 \).


\section{Билет 20}

\subsection*{Устойчивость знака непрерывной функции}

\textbf{Теорема:} Пусть $f$ — непрерывная функция на множестве $D \subset \mathbb{R}$, и пусть $c \in D$ такая точка, что $f(c) \neq 0$. Тогда существует окрестность $U(c)$ точки $c$, такая что для всех $x \in U(c) \cap D$ выполняется $f(x) \neq 0$ и знак функции $f$ на $U(c) \cap D$ совпадает со знаком $f(c)$.


\section{Билет 21}

\subsection*{1. Алгебраические функции}
Алгебраические функции — это функции, которые могут быть выражены с использованием конечного числа операций сложения, вычитания, умножения, деления и извлечения корней. Примеры:
\begin{itemize}
    \item линейная функция: $f(x) = ax + b$, где $a, b \in \mathbb{R}$;
    \item квадратичная функция: $f(x) = ax^2 + bx + c$, где $a, b, c \in \mathbb{R}$;
    \item корневая функция: $f(x) = \sqrt[n]{x}$, где $n \in \mathbb{N}, n \geq 2$.
\end{itemize}

\subsection*{2. Трансцендентные функции}
Трансцендентные функции не могут быть выражены в виде конечных комбинаций алгебраических операций. Они включают:
\begin{itemize}
    \item экспоненциальные функции, например, $f(x) = a^x$, где $a > 0, a \neq 1$;
    \item логарифмические функции, например, $f(x) = \ln(x)$ или $f(x) = \log_a(x)$;
    \item тригонометрические функции: $\sin(x)$, $\cos(x)$, $\tan(x)$ и т.д.;
    \item обратные тригонометрические функции: $\arcsin(x)$, $\arccos(x)$ и т.д.;
    \item гиперболические функции: $\sinh(x)$, $\cosh(x)$ и т.д.
\end{itemize}

\section{Билет 21}

\subsection*{3. Непрерывность элементарных функций}
Элементарные функции являются непрерывными на своих областях определения. Это означает, что если функция определена в некоторой точке $x_0$ и в её окрестности, то:
\[
\lim_{x \to x_0} f(x) = f(x_0).
\]
Примеры:
\begin{itemize}
    \item Линейные и квадратичные функции непрерывны на всей числовой прямой $\mathbb{R}$.
    \item Тригонометрические функции $\sin(x)$ и $\cos(x)$ непрерывны на $\mathbb{R}$, а $\tan(x)$ — на множестве $\mathbb{R} \setminus \left\{x = \frac{\pi}{2} + k\pi, \, k \in \mathbb{Z}\right\}$.
\end{itemize}

\section*{Операции над непрерывными функциями и переход к пределу под знаком непрерывной функции}

\subsection*{Операции над непрерывными функциями}
Пусть $f$ и $g$ — функции, непрерывные в точке $x = a$. Тогда следующие функции также непрерывны в точке $a$:
\begin{itemize}
    \item Сумма: $f(x) + g(x)$
    \item Разность: $f(x) - g(x)$
    \item Произведение: $f(x) \cdot g(x)$
    \item Частное: $\frac{f(x)}{g(x)}$, если $g(a) \neq 0$
\end{itemize}

\subsection*{Переход к пределу под знаком непрерывной функции}
Пусть $f$ — непрерывная функция в точке $a$, и пусть $\lim_{x \to a} g(x) = L$. Тогда:
\[
\lim_{x \to a} f(g(x)) = f\left(\lim_{x \to a} g(x)\right) = f(L)
\]

\textbf{Доказательство:} Так как $f$ непрерывна в точке $L$, то по определению непрерывности для любого $\epsilon > 0$ существует $\delta > 0$ такое, что для всех $y$, удовлетворяющих условию $|y - L| < \delta$, выполняется $|f(y) - f(L)| < \epsilon$. Поскольку $\lim_{x \to a} g(x) = L$, существует такое $\delta' > 0$, что для всех $x$, удовлетворяющих условию $|x - a| < \delta'$, выполняется $|g(x) - L| < \delta$. Следовательно, для таких $x$ имеем:
\[
|f(g(x)) - f(L)| < \epsilon
\]
Таким образом, $\lim_{x \to a} f(g(x)) = f(L)$.

\section{Билет 22}

\section*{Теорема о непрерывности сложной функции}

\textbf{Теорема:} Пусть $f$ непрерывна в точке $a$, и $g$ непрерывна в точке $b = f(a)$. Тогда сложная функция $h(x) = g(f(x))$ непрерывна в точке $a$.

\textbf{Доказательство:} Так как $f$ непрерывна в точке $a$, то для любого $\epsilon > 0$ существует $\delta_1 > 0$ такое, что если $|x - a| < \delta_1$, то $|f(x) - f(a)| < \delta_2$, где $\delta_2$ будет определено далее.

Поскольку $g$ непрерывна в точке $b = f(a)$, то для любого $\epsilon > 0$ существует $\delta_2 > 0$ такое, что если $|y - f(a)| < \delta_2$, то $|g(y) - g(f(a))| < \epsilon$.

Теперь, выберем $\delta = \delta_1$. Тогда, если $|x - a| < \delta$, то $|f(x) - f(a)| < \delta_2$, и, следовательно, $|g(f(x)) - g(f(a))| < \epsilon$.

Таким образом, $|h(x) - h(a)| = |g(f(x)) - g(f(a))| < \epsilon$, что доказывает непрерывность $h(x)$ в точке $a$.


\section{Билет 23}
\subsection*{Первый замечательный предел}
\[
\lim_{x \to 0} \frac{\sin x}{x} = 1
\]

\textbf{Следствия:}
\[
\lim_{x \to 0} \frac{\tan x}{x} = 1, \quad \lim_{x \to 0} \frac{\arcsin x}{x} = 1, \quad \lim_{x \to 0} \frac{\arctan x}{x} = 1
\]
\[
\lim_{x \to 0} \frac{1 - \cos x}{x^2/2} = 1
\]

\subsection*{Второй замечательный предел}
\[
\lim_{x \to \infty} \left(1 + \frac{1}{x}\right)^x = e
\]

\textbf{Следствия:}
\[
\lim_{x \to 0} (1 + x)^{1/x} = e
\]
\[
\lim_{k \to +\infty} \left(1 + \frac{1}{k}\right)^k = e
\]
\[
\lim_{x \to 0} \ln(1 + x) = 1
\]
\[
\lim_{x \to 0} \frac{e^x - 1}{x} = 1
\]
\[
\lim_{x \to 0} \frac{a^x - 1}{x} = \ln(a) \quad \text{для} \quad a > 0, \, a \neq 1
\]
\[
\lim_{x \to 0} \frac{\ln(1 + ax)}{ax} = 1
\]


\end{document}
